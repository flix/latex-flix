\documentclass{article}

\usepackage{inconsolata}
\usepackage{microtype}

\usepackage{flix}

\lstset{language=flix}

\begin{document}

\section{Algebraic Data Types and Pattern Matching}

\begin{lstlisting}
/// An algebraic data type for shapes.
enum Shape {
    case Circle(Int32),          // circle radius
    case Square(Int32),          // side length
    case Rectangle(Int32, Int32) // height and width
}

/// Computes the area of the given shape using
/// pattern matching and basic arithmetic.
def area(s: Shape): Int32 = match s {
    case Shape.Circle(r)       => 3 * (r * r)
    case Shape.Square(w)       => w * w
    case Shape.Rectangle(h, w) => h * w
}

// Computes the area of a 2 by 4.
def main(): Unit \ IO =
    println(area(Shape.Rectangle(2, 4)))
\end{lstlisting}

\section{Lists and List Processing}

\begin{lstlisting}
/// In Flix, as in many functional programming languages, 
/// lists are the bread and butter.

/// We can easily construct and append lists:
def l(): List[Int32] =
    let l1 = 1 :: 2 :: 3 :: Nil;
    let l2 = 4 :: 5 :: 6 :: Nil;
    l1 ::: l2

/// We can use pattern matching to take a list apart:
def length(l: List[a]): Int32 = match l {
  case Nil     => 0
  case _ :: xs => 1 + length(xs)
}

/// The Flix library has extensive support for lists:
def main(): Unit \ IO =
    let l1 = l();
    let l2 = List.intersperse(42, l1);
    let l3 = List.map(x -> x :: x :: Nil, l2);
    let l4 = List.flatten(l3);
    println(length(l4))
\end{lstlisting}

\end{document}
